\documentclass{beamer}
\usetheme{metropolis}           % Use metropolis theme
\setsansfont[BoldFont={fira-sans.bold.ttf},ItalicFont={fira-sans.light-italic.ttf}]{fira-sans.light.ttf}
\setmonofont{fira-mono.regular.ttf}
\setbeameroption{show notes}

% PACKAGES %
\usepackage{amsmath}
\usepackage{amssymb}
\usepackage{pifont}
\usepackage{tikz}
\usetikzlibrary{calc, trees, fit, positioning}
\newfontfamily\DejaSans{DejaVu Sans}
\usepackage{booktabs}
\usepackage{siunitx}
\usepackage{diagbox}
\usepackage{multirow}

% COLORS %
\definecolor{Yes}{HTML}{0F771D}
\definecolor{Maybe}{HTML}{1F549E}
\definecolor{No}{HTML}{9F1A11}

% SYMBOLS %
\newcommand{\xmark}{\ding{55}}

\title{Learning High-Order Word Representations}
\date{June 12, 2018}
\author{Konstantinos Kogkalidis}
\institute{LoLa Fan Club}

\metroset{
titleformat=regular, 
sectionpage=progressbar, 
subsectionpage=progressbar, 
progressbar=frametitle,
titleformat frame=regular,
block=fill}


\begin{document}
  \maketitle
 
\section{Motivation}
\begin{frame}{Distributional Compositional Semantics}

	Idea: structure-preserving map $\mathcal{F}$
	\[
	\mathcal{F}: \mathcal{G} \to \pmb{FdVect} 
	\]
	
	\begin{itemize}	
	\item Atomic types translated to vectors (order-one tensors)
	\item Complex types translated to (multi-)linear maps (higher order tensors)
	\end{itemize}
\end{frame}

\begin{frame}{Example}
\centering

	\vspace{40pt}
	\begin{tabular}{c|c|c}
	\toprule
	Word Type & $\mathcal{G}$ Type & $\mathcal{F}$ Translation\\
	\midrule
	Noun & $NP$ & $\mathbb{R}^{NP}$ \\ 
	Adjective & $NP / NP$ & $\mathbb{R}^{NP \times NP} \equiv \mathbb{R}^{NP} \to \mathbb{R}^{NP}$\\
	\end{tabular}
	
	\vspace{20pt}
	\begin{align*}
	\pmb{cat} & \in \mathbb{R}^{NP}\\
	\pmb{black},\ \pmb{stray} & \in \mathbb{R}^{NP \times NP}\\
	\pmb{black\ stray\ cat} & \in \mathbb{R}^{NP}\\
	\end{align*}	
	
\end{frame}
  
\begin{frame}{Why Compositionality?}
	\begin{itemize}
	\item Bridging of formal \& distributional semantics
	\item Syntax-informed meaning derivations
	\item Modeling of functional words
	\item Formal treatment of ambiguous derivations
	\item Contextual Disambiguation
	\item Richer representations
	\item[] \quad \vdots
	\end{itemize}
\end{frame}

\begin{frame}{Why Not Compositionality?}
	\begin{itemize}
	\item[\textcolor{Yes}{\checkmark}] Great properties
	\item[\textcolor{Maybe}{\textbf{?}}] How to obtain word representations?
	\end{itemize}
	
	
	Possible options:
	\begin{enumerate}
	\pause
	\item Co-occurrence statistics 
	\pause \textcolor{No}{\xmark}
	\item Unsupervised techniques (\textit{a la word2vec})
	\pause \textcolor{No}{\xmark}
	\item Supervised learning \textcolor{Maybe}{\textbf{?}}
	\end{enumerate}
\end{frame}

\begin{frame}{Problem Statement}
\centering
Examine whether supervised learning can be used to find higher-order word representations (transitive verbs)
\end{frame}
	
\section{Supervised Learning}

\begin{frame}{Overview}
	\begin{itemize}
	\item Search over set of functions $A \to B$ parameterized over $P$
	\item Find optimal approximation $\hat{f}_P$ to $f: A \to B$
	\item Use samples $(a, f(a)) \in A\times B$ to update $P$
	\end{itemize}	
\end{frame}

\subsection{Dataset} 

\begin{frame}{Finding Data}
	Sample space must be:
	\begin{itemize}
	\item labeled
	\item constrained
	\item of large size
	\item of high quality
	\end{itemize}
\end{frame}

\begin{frame}{Paraphrase Database}

	Raw text paraphrase pairs
	
	
	\begin{block}{Example pair}
	\small \textit{proposed by the president} $\sim$ \textit{suggested by the chairman}
	\end{block}
	
	\begin{itemize}
	\item labeled \textcolor{Yes}{\checkmark}
	\item constrained \textcolor{No}{\xmark}
	\footnotesize\textit{(different syntactic types)}
	\normalsize
	\item of large size \textcolor{Yes}{\checkmark}
	\item of high quality \textcolor{Maybe}{\textbf{?}}
	\end{itemize}
\end{frame}

\begin{frame}{Dataset: Preprocessing}

	\begin{enumerate}
	\item \textbf{Parse and filter by type} (transitive verb case)
	\end{enumerate}

	\begin{itemize}
	\item labeled \textcolor{Yes}{\checkmark}
	\item constrained \textcolor{Yes}{\checkmark}
	\item of large size \textcolor{No}{\xmark} 
	\footnotesize\textit{(>95\% loss)} 
	\normalsize
	\item of high quality \textcolor{No}{\xmark}
	\footnotesize\textit{(parser-induced errors)} 
	\normalsize
	\end{itemize}
\end{frame}

\begin{frame}{Dataset: Preprocessing}

	\begin{enumerate}
	\item[2.] \textbf{Back-translation}
	\end{enumerate}

	\begin{itemize}
	\item Labeled \textcolor{Yes}{\checkmark}
	\item constrained \textcolor{Yes}{\checkmark}
	\item of large size \textcolor{Yes}{\checkmark}
	\item of high quality \textcolor{No}{\xmark}
	\footnotesize\textit{(translation-induced errors)}
	\normalsize
	\end{itemize}
\end{frame}

\begin{frame}{Dataset: Preprocessing}

	\begin{enumerate}
	\item[3.] \textbf{Filter by co-occurrence / mutual information}
	\end{enumerate}

	\begin{itemize}
	\item Labeled \textcolor{Yes}{\checkmark}
	\item constrained \textcolor{Yes}{\checkmark}
	\item of large size \textcolor{Yes}{\checkmark}
	\item of high quality \textcolor{Maybe}{\textbf{?}}
	\end{itemize}
\end{frame}
    
\begin{frame}{Dataset: End Result}
	Verb / object dictionaries:
	\begin{align*}
	\mathcal{V}&: \{v_1: 1, v_2: 2, ..., v_N: N\}\\
	\mathcal{O}&: \{o_1: 1, o_2: 2, ..., o_M: M\}
	\end{align*}
	Paraphrase relation: 
	\begin{align*}
	\mathcal{P}: \mathbb{N} \times \mathbb{N} \times \mathbb{N} \times \mathbb{N} &\to \{0, 1\} \tag{\alert{binary classification}}\\
	\mathcal{P}(i,j,k,l) = \mathcal{P}(k,l,i,j)
	 &= \begin{cases}
	1 & v_i o_j \sim v_k o_l\\
	0 & \text{otherwise}\\
	\end{cases}
	\end{align*}	
\end{frame}
  
\subsection{Formulating the Network}

  \begin{frame}{Training Objective}
    Our semantic interpretations are:
  	\begin{itemize}
  	\item Actions: $ \lceil a \rceil = \mathbb{R}^{A}$
  	\item Objects: $ \lceil np \rceil = \mathbb{R}^{NP}$
  	\item Transitive Verbs: $\lceil a / np \rceil = \mathbb{R}^{A\times NP}$
  	\end{itemize}
  
  	And our objective is to learn a \alert{verb embedding function} 
  	$\varepsilon_{verb}$:
  	\[
  	\varepsilon_{verb}: \mathbb{N} \to \mathbb{R}^{A\times NP}
  	\]
  	
  	But instead we have samples from some $f: \mathbb{N}^4 \to \{0,1\}$ 
  	

  	
\end{frame}

\begin{frame}{Intermediate Representations}
  	\begin{block}{Solution}
	Formulate $f_P$ to incorporate $\varepsilon_{verb}$.
	\[
	f_p = f_1 \circ f_2 \circ \dots \circ \varepsilon_{verb} \circ \dots
	\] 
  	\end{block}
  	 
  	\pause
  	\begin{block}{Simplification (1)}
  	Assume pre-trained \alert{object embedding function} $\varepsilon_{object}$.
  	\[
  	\varepsilon_{object}: \mathbb{N} \to \mathbb{R}^{300}
  	\]
  	\end{block}
\end{frame}

\begin{frame}{Filling the missing blocks}
\begin{figure}
\begin{tikzpicture}
	[auto,
	scale=0.8,
	item/.style = {rectangle, text width = 6em, align=center},
	function/.style = {->}, 
	known/.style = {gray!90}]
						
	\foreach \x/\alph/\name in {
			0/n11/ \small $i \in \mathbb{N}$,
		  	3/n12/ \small $j \in \mathbb{N}$,
		  	6/n13/ \small $k \in \mathbb{N}$,
		  	9/n14/ \small $l \in \mathbb{N}$
		 }{
		 	\draw (\x,0) node[item] (\alph) {\textcolor{mDarkTeal}{\name}};
		 }
		 
	\draw (4.5, -9) node[item] (n51) {\textcolor{mDarkTeal}{$\hat{y} \in \mathbb{R}$}};	
	
	
	% object embeddings
	\pause 
	\foreach \x/\alph/\name in {
			3/n22/ \small $\pmb{j} \in \mathbb{R}^{300}$,
		  	9/n24/ \small $\pmb{l} \in \mathbb{R}^{300}$
		 }{
		 	\draw (\x,-3) node[item] (\alph) {\textcolor{mDarkTeal}{\name}};
		 }

		\foreach \source/\target in {n12/n22,n14/n24}
	{
		\draw (\source) edge[known, ->, edge label=$\varepsilon_{object}$] (\target);
	}

	% verb embeddings
	\pause
	\foreach \x/\alph/\name in {
			6/n23/ \small $\pmb{\mathcal{K}} \in \mathbb{R}^{100 \times 300}$,			
			0/n21/ \small $\pmb{\mathcal{I}} \in \mathbb{R}^{100 \times 300}$,		}{
			\draw (\x, -3) node[item] (\alph) {\textcolor{mDarkTeal}{\name}};
		 }
		 
	\foreach \source/\target in {n11/n21,n13/n23}
	{
		\draw (\source) edge[function, edge label=$\varepsilon_{verb}$] (\target);
	}
	
	% phrase embeddings
	\pause
	\foreach \x/\alph/\name in {
			1.5/n31/ \small $\pmb{\mathcal I}(\pmb{j}) \in \mathbb{R}^{100}$,
		  	7.5/n32/ \small $\pmb{\mathcal K}(\pmb{l}) \in \mathbb{R}^{100}$,
		 }{
		 	\draw (\x,-6) node[item] (\alph) {\textcolor{mDarkTeal}{\name}};
		 }

	\draw (n22) edge[-, bend right=30] (n21);
	\draw (n21) edge[->, bend right=20] (n31.north);
	\draw (n24) edge[-, bend right=30] (n23);
	\draw (n23) edge[->, bend right=20] (n32.north);	
	
	\pause
	\draw (4.5, -7.5) node[item] (n41) {};

	\draw (n31) edge[known, bend right=20] (n41.center);
	\draw (n32) edge[known, bend left=20] (n41.center);
	\draw (n41.center) edge[known,->, near start, edge label= \small $cosine$] (n51);	
\end{tikzpicture}
\end{figure}
\end{frame}

\begin{frame}{Sanity Check}
\begin{block}{Objective Function}
\begin{align*}
cos(\pmb{V}_i(\pmb{j}), \pmb{V}_k(\pmb{l})) \rightsquigarrow \mathcal{P}(i,j,k,l) 
\quad
\forall \ (i,j,k,l) \in \mathcal{V} \times \mathcal{O} \times \mathcal{V} \times \mathcal{O}
\end{align*}
\end{block}

\pause
Considerations:
\begin{enumerate}
\item Network Size\\
\begin{itemize}
\item $1.000$ verbs
\item $100 \times 300 = 30.000$ parameters per verb
\item[$\Rightarrow$]  \alert{ $ 30.000.000 $ parameters} for $\varepsilon_{verb}$ to learn
\end{itemize}
\pause
\item Quantifying over \alert{two spaces} ...
\pause
\item ... both of which are \alert{non-convex}
\end{enumerate}
\end{frame}

\begin{frame}[standout]
\includegraphics[scale=0.25]{beast.jpg}

... A "beast" to train {\DejaSans ☹} 
\end{frame}


\subsection{Transferring Knowledge}

\begin{frame}{Finding an Oracle}
Dataception: use our labeled dataset to create a new dataset

\pause
\begin{block}{Simplification (2)}
Assume another pre-trained \alert{verb embedding function} $\varepsilon '_{verb}$.
  	\[
  	\varepsilon '_{verb}: \mathbb{N} \to \mathbb{R}^{300}
  	\]
\end{block}

\pause
\begin{block}{Oracle}
We can now train a \alert{paraphrase embedding function} $\varepsilon_{par}$.
	\[
	\varepsilon_{par}: \mathbb{R}^{300} \times \mathbb{R}^{300} \to \mathbb{R}^{100}
	\]
\end{block}
\end{frame}

\begin{frame}{Oracle Flow}
\begin{figure}
\begin{tikzpicture}
	[auto,
	scale=0.8,
	item/.style = {rectangle, text width = 6em, align=center},
	function/.style = {->}, 
	known/.style = {gray!90}]
	
	% I/O					
	\foreach \x/\alph/\name in {
			0/n11/ \small $i \in \mathbb{N}$,
		  	3/n12/ \small $j \in \mathbb{N}$,
		  	6/n13/ \small $k \in \mathbb{N}$,
		  	9/n14/ \small $l \in \mathbb{N}$
		 }{
		 	\draw (\x,0) node[item] (\alph) {\textcolor{mDarkTeal}{\name}};
		 }
		 
	\draw (4.5, -9) node[item] (n61) {\small $\hat{y} \in \mathbb{R}$};
	
	% object embeddings
	\pause 
	\foreach \x/\alph/\name in {
		  	3/n22/ \small $\pmb{j} \in \mathbb{R}^{300}$,
		  	9/n24/ \small $\pmb{l} \in \mathbb{R}^{300}$
		 }{
		 	\draw (\x,-3) node[item] (\alph) {\textcolor{mDarkTeal}{\name}};
		 }

	\foreach \source/\target in {n12/n22,n14/n24}
	{
		\draw (\source) edge[known, ->, edge label=$\varepsilon_{object}$] (\target);
	}
	
	% verb embeddings
	\pause 
	\foreach \x/\alph/\name in {
			0/n21/ \small $\pmb{i} \in \mathbb{R}^{300}$,
			6/n23/ \small $\pmb{k} \in \mathbb{R}^{300}$,			
		}{
		 	\draw (\x,-3) node[item] (\alph) {\textcolor{mDarkTeal}{\name}};
		 }
	
	\foreach \source/\target in {n11/n21,n13/n23}
	{
		\draw (\source) edge[known, ->, edge label=$\varepsilon 
		'_{verb}$] (\target);
	}
				
	% phrase embeddings
	\pause				 
	\foreach \x/\alph in {
			1.5/n31,
		  	7.5/n32,
		 }{
		 	\draw (\x,-4.5) node[item] (\alph) {};
		 }
	
	\foreach \x/\alph/\name in {
			1.5/n41/ \small $\pmb{m} \in \mathbb{R}^{100}$,
			7.5/n42/ \small $\pmb{n} \in \mathbb{R}^{100}$
		}{
			\draw (\x, -6) node[item] (\alph) {\textcolor{mDarkTeal}{\name}};
		}
	
	\draw (4.5, -7.5) node[item] (n51) {};
		 
	\draw (n21.south) edge[-, bend right = 36] (n31.center);
	\draw (n22.south) edge[-, bend left = 36] (n31.center);	
	\draw (n31.center) edge[function, edge label=$\varepsilon_{par}$] (n41);
	
	\draw (n23.south) edge[-, bend right = 36] (n32.center);
	\draw (n24.south) edge[-, bend left = 36] (n32.center);	
	\draw (n32.center) edge[function, edge label=$\varepsilon_{par}$] (n42);
	
	\pause
	\draw (n41) edge[known, bend right=20] (n51.center);
	\draw (n42) edge[known, bend left=20] (n51.center);
	\draw (n51.center) edge[known, ->, near start, edge label= \small $cosine$] (n61);	
\end{tikzpicture}
\end{figure}
\end{frame}

\begin{frame}{Training the Oracle}
$\varepsilon_{par}$: \alert{recurrent autoencoder} ($\approx 700.000$ parameters)
\includegraphics[scale=0.25]{N1.png}
\end{frame}

\begin{frame}{Utilizing the Oracle}
\begin{block}{New Objective Function}
\[
cos(\pmb{V}_i(\pmb{j}), \varepsilon_{par}(\pmb{v}_i,\pmb{j})) \rightsquigarrow 1
\quad \forall \ (i,j) \in \mathcal{V} \times \mathcal{O}
\]
\end{block}

\begin{itemize}
\item $\varepsilon_{par}$ gives us paraphrase embeddings '\textit{for free}'
\item We can use them to facilitate training
\item Much smaller problem space
\end{itemize}
\end{frame}

\begin{frame}{Composing Networks}
\begin{figure}
\begin{tikzpicture}
	[scale=0.3,
	item/.style = {rectangle, text width = 6em, align=center},
	function/.style = {->}, 
	known/.style = {gray!90}]
	
	\foreach \x/\alph/\name in {
			0/n11/ \small $i$,
		  	6/n12/ \small $j$,
		  	12/n13/ \small $k$,
		  	18/n14/ \small $l$
		 }{
		 	\draw (\x,0) node[item] (\alph) {\textcolor{mDarkTeal}{\name}};
		 }
		 
	\foreach \x/\alph/\name in {
			-6/n21/ \small $\pmb{I}$,
		  	0/n22/ \small $\pmb{i}$,
		  	6/n23/ \small $\pmb{j}$,
		  	12/n24/ \small $\pmb{K}$,
		  	18/n25/ \small $\pmb{k}$,
		  	24/n26/ \small $\pmb{l}$
		 }{
		 	\draw (\x,-6) node[item] (\alph) {\textcolor{mDarkTeal}{\name}};
		 }
		 
	\foreach \x/\alph in {3/n31, 21/n32}{
		\draw (\x, -8) node (\alph) {};
	}
	
	\foreach \x/\alph/\name in {
		-6/n41/$\pmb{I}(\pmb{j})$, 3/n42/$\pmb{m}$, 12/n43/$\pmb{K}(\pmb{l})$, 21/n44/$\pmb{n}$}{
		\draw (\x, -10) node[item] (\alph) {\textcolor{mDarkTeal}{\name}};
	}
	
	\foreach \x/\alph in 
	{-2/n51, 16.5/n52}
	{\draw (\x, -12) node (\alph) {};}
	
	\foreach \x/\alph/\name in {
		-2/n61/$\hat{y} '$, 16.5/n62/$\hat{y} ''$}
	{\draw (\x, -14) node (\alph) {\textcolor{mDarkTeal}{\name}};}

	\draw (3, -18) node (n71) {};
	\draw (3, -21) node (n81) {$\hat{y}$};
	
	\foreach \source/\target in {n23/n21, n26/n24}
	{\draw (\source) edge[bend right=40] (\target);}
		
		 
	\foreach \source/\target in {n11/n22, n12/n23, n13/n25, n14/n26}{
		\draw (\source) edge[known, ->] (\target);
	}
	
	\foreach \source/\target in {n11/n21, n13/n24}{
		\draw (\source) edge[->] (\target);
	}
	
	\foreach \source/\target in {n22/n31,  n25/n32}
	{\draw (\source) edge[known, bend right=25] (\target.center);}
	\foreach \source/\target in {n23/n31,  n26/n32}
	{\draw (\source) edge[known, bend left=25] (\target.center);}
	\foreach \source/\target in {n31/n42, n32/n44}
	{\draw (\source.center) edge[known, function] (\target);}
	\foreach \source/\target in {n42/n51, n44/n52}
	{\draw (\source) edge[known, bend left=15] (\target.center);}
	\foreach \source/\target in {n21/n41, n24/n43}
	{\draw (\source) edge[function] (\target);}
	\foreach \source/\target in {n41/n51, n43/n52}
	{\draw (\source) edge[known, bend right=15] (\target.center);}
	\foreach \source/\target in {n51/n61, n52/n62}
	{\draw (\source.center) edge[known, ->] (\target);}
	\draw (n41) edge[known, bend right=40] (n71.center);
	\draw (n43) edge[known, bend left=40] (n71.center);
	\draw (n71.center) edge[known, function] (n81);	
\end{tikzpicture}
\end{figure}
\end{frame}

\begin{frame}{Training the Original}
$\varepsilon_{verb}$: \alert{tanh activated dense layer} ($\approx 30.000.000$ parameters)
\includegraphics[scale=0.25]{N2.png}
\end{frame}

\begin{frame}[standout]
\includegraphics[scale=0.25]{beauty2.jpg}\\
The beast has been tamed! {\DejaSans ☺} 
\end{frame}

\section{Evaluation}

\begin{frame}{Microstructure}
Task-specific performance relates to the small-scale structure of the learned space:
\vspace{10pt}

\begin{tabular}{c|SS|SS} \toprule
    {\multirow{2}{*}{\diagbox{\scriptsize Ground Truth}{\scriptsize Prediction}}} & 
    \multicolumn{2}{S}{$\text{Oracle}$} & 
    \multicolumn{2}{S}{$\text{Final}$}\\
    & $\top$ & $\bot$ & $\top$ & $\bot$\\
    \midrule
    $\top$  & \color{Yes}{0.92} & \color{No}{0.08} & \color{Yes}{0.88} & \color{No}{0.02} \\
    $\bot$  & \color{No}{0.08} & \color{Yes}{0.92} & \color{No}{0.12} & \color{Yes}{0.98} \\
   \bottomrule
\end{tabular}
\end{frame}

\begin{frame}{Macrostructure}
\alert{3D PCA on paraphrase embeddings}

\includegraphics[scale=0.29]{mapper.png}
\end{frame}

\section{Conclusion}

\begin{frame}{Critique}
\begin{enumerate}
\item Metric reliability
\item Uninformative error signal
\item Over-parameterization
\begin{itemize}
\item[a)] Linearity constraint
\item[b)] Chasing after an oracle
\item[c)] Bad scaling
\end{itemize}
\end{enumerate}
\end{frame}

\begin{frame}{Next Steps}
\begin{enumerate}
\item Directly evaluate verb matrices
\item More structural constraints (activity regularization)
\item Iterative learning
\item Other data formats:
\begin{itemize}
\item Different syntactic types
\item Different labels / samples altogether
\end{itemize}
\item Different oracle architectures
\item Different embedder architectures (encoder/decoder)
\end{enumerate}
\end{frame}

\section{Bag of Tricks}
\begin{frame}{Negative Sampling}
Let $L: \mathbb{R} \times \mathbb{R} \to \mathbb{R}$ be the \textit{loss function}. 

\vspace{20pt}
Objective translates to:
\begin{align}
\tag{\textcolor{No}{Intractable}}
\min_{P}\  L\left[\mathcal{P}\left( i,j,k,l\right),\ \hat{y}_P\right]
\quad \forall (i,j,k,l)
\end{align}

Randomly generate and select negative samples (different every epoch).
\end{frame}

\begin{frame}{Curriculum Learning}
\vspace{20pt}
\centering
"\textit{Two phrases are not similar unless they are}"

\vspace{30pt}
\begin{itemize}
\item Class imbalance $\Rightarrow$ bad predictions
\item Treat negative samples as noise
\item Learn positive examples then increase noise
\end{itemize}
\end{frame}

\begin{frame}{Loss Mixing}
\vspace{20pt}
\centering
Cross-entropy vs. MSE vs. Categorical Hinge

\vspace{30pt}
\begin{itemize}
\item Different assumptions, none correct.
\item Many falsities $\Rightarrow$ Truth \textcolor{Maybe}{\textbf{?}}
\end{itemize}
\end{frame}
\end{document}
